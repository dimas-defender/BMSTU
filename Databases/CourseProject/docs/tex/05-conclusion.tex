\section*{ЗАКЛЮЧЕНИЕ}
\addcontentsline{toc}{section}{ЗАКЛЮЧЕНИЕ}

В ходе выполнения курсовой работы была достигнута поставленная цель -- разработана база данных магазина электроники и приложение доступа к ней. В процессе работы были решены следующие задачи:

\begin{itemize}[leftmargin=0.7cm + \labelwidth - \labelsep]
	\item[---] формализованы задача и данные;
	\item[---] проанализированы типы СУБД по модели данных;
	\item[---] проведен обзор существующих аналогов;
	\item[---] описана структура базы данных;
	\item[---] создана база данных с ролевой моделью;
	\item[---] спроектирован интерфейс для доступа к БД;
	\item[---] разработано приложение для взаимодействия с созданной БД;
	\item[---] исследовано влияние использования индексов на время выполнения запросов к базе данных.
\end{itemize}

Были закреплены знания по теории баз данных, получены практические навыки проектирования и разработки баз данных и приложений для взаимодействия с ними.

В результате исследования было установлено, что время выполнения запросов линейно зависит от количества строк в таблицах с данными. Сделан вывод о том, что использование индексов для столбцов, по значениям которых часто осуществляется поиск данных, позволяет значительно повысить его скорость.

\pagebreak