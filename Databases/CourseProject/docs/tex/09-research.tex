\section{Исследовательский раздел}

\subsection{Описание исследования}

Цель исследования -- сравнить время выполнения запросов к базе данных с использованием индексов и без него.

Для этого необходимо составить запрос, выбирающий значения из одной из таблиц базы данных, замерить время его выполнения, затем создать индекс для столбца этой таблицы, по которому происходит поиск значений, и повторно замерить время.

\textbf{Постановка эксперимента}

Индекс — это объект базы данных, позволяющий повысить производительность поиска данных. Индекс представляет собой упорядоченный файл с последовательностью пар ключей и указателей.

Без использования индексов поиск данных по значению определенных столбцов сводится к последовательному просмотру таблицы строка за строкой, что потребует значительных временных затрат при достаточных размерах таблиц. В индексе же данные хранятся в форме, позволяющей в процессе поиска не рассматривать области, которые заведомо не могут содержать искомые элементы, что повышает скорость поиска.

Для проведения эксперимента используется запрос на получение количества товаров, находящихся в определенном ценовом диапазоне. Поиск строк ведется в таблице товаров по значению цены.

Замеряется среднее время выполнения $100$ таких запросов в миллисекундах, для чего запрос выполняется $1000$ раз. Перед второй серией замеров добавляется индекс для поля цены таблицы товаров.

Замеры времени проводятся последовательно на размерах таблиц в $1000$, $2000$, $5000$ и $10000$ строк, заполненных случайно сгенерированными данными.

\pagebreak

\textbf{Скрипт для проведения замеров времени}

Скрипт для проведения замеров времени выполнения запросов приведен в листинге \ref{research:qtime}.

\lstinputlisting[
firstline=1,
lastline=31,
caption={Замер времени выполнения запросов к базе данных},
label={research:qtime}
]
{listings/research.py}

\pagebreak

\textbf{Технические характеристики}

Ниже приведены технические характеристики устройства, на котором было проведено исследование:

\begin{itemize}[leftmargin=0.7cm + \labelwidth - \labelsep]
	\item[---] операционная система Windows 10 64-разрядная;
	\item[---] оперативная память 16 ГБ;
	\item[---] процессор Intel(R) Core(TM) i5-4690 @ 3.50ГГц.
\end{itemize}

Во время эксперимента компьютер был подключен к сети электропитания и нагружен исключительно исследованием.

\subsection{Результаты исследования}

Результаты эксперимента отражены в таблице \ref{research:results}. Показано среднее время выполнения 100 запросов в миллисекундах.

\begin{table}[H]
	\caption{Результаты исследования}
	\label{research:results}
	\small
	\begin{tabular}{|c|c|c|}
		\hline
		\begin{tabular}[c]{@{}c@{}}Используется\\индекс\end{tabular} & \begin{tabular}[c]{@{}c@{}}Количество строк\\ в таблицах,  тыс. шт\end{tabular} & Время выполнения, мс \\ \hline
		\multirow{4}{*}{Нет} & 1 & 20.5 \\ \cline{2-3}
		& 2 & 29.7 \\ \cline{2-3}
		& 5 & 56.2 \\ \cline{2-3}
		& 10 & 93.7 \\ \hline
		\multirow{4}{*}{Да} & 1 & 14.1 \\ \cline{2-3} 
		& 2 & 15.6 \\ \cline{2-3} 
		& 5 & 21.9 \\ \cline{2-3} 
		& 10 & 31.8 \\ \hline
	\end{tabular}
\end{table}

\pagebreak

Зависимость времени выполнения $100$ запросов от размера таблиц представлена на рисунке \ref{research:graph}.

\begin{figure}[H]
	\centering
	\begin{tikzpicture}[scale=1.7]
		\begin{axis}[
			axis lines=left,
			ylabel style={align=center}, ylabel={Время, мс},
			legend pos=north west,
			ymajorgrids=true,
			xlabel style={align=center}, xlabel={Количество строк\\в таблицах, тыс. шт},
			]
			\addplot table[x=N,y=ms,col sep=comma]{csv/index.csv};
			\addplot table[x=N,y=ms,col sep=comma]{csv/noindex.csv};
			\legend{{С индексом}, {Без индекса}}
		\end{axis}
	\end{tikzpicture}
	%\captionsetup{justification=centering}
	\caption{Зависимость времени выполнения 100 запросов от размера таблиц}
	\label{research:graph}
\end{figure}

По полученным результатам можно сделать вывод, что время выполнения запросов линейно увеличивается с ростом числа строк в таблицах базы данных. При этом использование индекса дает выигрыш по времени уже на малых размерах таблиц, а также обеспечивает более медленный рост временных затрат при увеличении объемов данных. Так, при увеличении числа строк таблиц с 1000 до 10000 время выполнения запросов без использования индекса увеличилось в 4.6 раза, а с использованием индекса -- лишь в 2.3 раза.

\pagebreak

\subsection*{Вывод}

В данном разделе было исследовано влияние использования индексов на время выполнения запросов к базе данных. По результатам эксперимента можно сделать вывод о том, что целесообразно использовать индексы для столбцов, по значениям которых часто осуществляется поиск строк в таблицах. Использование индексов позволяет значительно повысить скорость поиска данных.

\pagebreak