\section*{СПИСОК ИСПОЛЬЗОВАННЫХ ИСТОЧНИКОВ}
\addcontentsline{toc}{section}{СПИСОК ИСПОЛЬЗОВАННЫХ ИСТОЧНИКОВ}

\begingroup
\renewcommand{\section}[2]{}
\begin{thebibliography}{}
	
	\bibitem{subd}
	Что такое СУБД [Электронный ресурс]. Режим доступа: https://www.nic.ru/help/chto-takoe-subd\_8580.html (дата обращения: 10.05.2022).
	
	\bibitem{models}
	Модели баз данных [Электронный ресурс]. Режим доступа: https://www.internet-technologies.ru/articles/modeli-baz-dannyh-sistemy-upravleniya-bazami-dannyh.html (дата обращения: 10.05.2022).
	
	\bibitem{123}
	Интернет-магазин техники и электроники <<123>> [Электронный ресурс]. Режим доступа: https://www.123.ru/ (дата обращения: 12.05.2022).
	
	\bibitem{citilink}
	Интернет-магазин техники и электроники <<Ситилинк>> [Электронный ресурс]. Режим доступа: https://www.citilink.ru/ (дата обращения: 12.05.2022).
	
	\bibitem{ligabt}
	Интернет-магазин техники и электроники <<Лига-БТ>> [Электронный ресурс]. Режим доступа: https://ligabt.ru/ (дата обращения: 12.05.2022).
	
	\bibitem{formulatv}
	Интернет-магазин техники и электроники <<FormulaTV>> [Электронный ресурс]. Режим доступа: https://formulatv.ru/ (дата обращения: 12.05.2022).
	
	\bibitem{oracle}
	Oracle [Электронный ресурс]. Режим доступа: https://www.oracle.com/ (дата обращения: 10.05.2022).
	
	\bibitem{mysql}
	MySQL [Электронный ресурс]. Режим доступа: https://www.mysql.com/ (дата обращения: 10.05.2022).
	
	\bibitem{sqlserver}
	Microsoft SQL Server [Электронный ресурс]. Режим доступа: https://www.microsoft.com/ru-ru/sql-server (дата обращения: 10.05.2022).
	
	\bibitem{postgresql}
	PostgreSQL [Электронный ресурс]. Режим доступа: https://www.postgresql.org/ (дата обращения: 10.05.2022).
	
\end{thebibliography}
\endgroup

\pagebreak