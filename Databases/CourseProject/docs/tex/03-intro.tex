\section*{ВВЕДЕНИЕ}
\addcontentsline{toc}{section}{ВВЕДЕНИЕ}

В современном мире значительная часть покупок совершается в интернет-магазинах. Количество клиентов онлайн-сервисов растет, а в эпоху информационных технологий электронная техника особенно популярна. Онлайн-магазины позволяют покупателю быстро и легко ознакомиться с интересующими его товарами, сделать свой выбор и оформить заказ с доставкой на дом.

Для хранения информации о товарах, поставщиках, продажах и клиентах онлайн-магазинов удобно использовать базы данных. Для взаимодействия с таким хранилищем данных необходим удобный интерфейс, который даст сотрудникам магазина возможность просматривать, редактировать и добавлять информацию, связанную с деятельностью предприятия, а покупателям – возможность комфортно осуществлять заказы.

Цель данной работы – разработать базу данных для организации работы магазина электроники, а также приложение для взаимодействия с ней.

Для достижения поставленной цели требуется решить следующие задачи:

\begin{itemize}[leftmargin=0.7cm + \labelwidth - \labelsep]
	\item[---] формализовать задачу, данные;
	\item[---] проанализировать типы СУБД;
	\item[---] провести обзор существующих аналогов;
	\item[---] описать структуру базы данных;
	\item[---] создать базу данных с ролевой моделью;
	\item[---] спроектировать интерфейс для доступа к БД;
	\item[---] разработать приложение для взаимодействия с созданной БД;
	\item[---] исследовать влияние использования индексов на время выполнения запросов к базе данных.
\end{itemize} 

\pagebreak