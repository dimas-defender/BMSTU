\section*{РЕФЕРАТ}
\addcontentsline{toc}{section}{РЕФЕРАТ}

Расчетно-пояснительная записка 37 с., \totalfigures\ рис., \totaltables\ табл., 10 источн., 1 прил.

Ключевые слова: база данных, доступ к данным, система управления базами данных, индекс, таблица, товар, пользователь, приложение, магазин.

Объектом разработки является база данных интернет-магазина электроники.

Область применения -- задача организации работы магазина, осуществляющего розничную торговлю.

Цель работы -- разработать базу данных для организации работы магазина электроники, а также приложение для взаимодействия с ней.

Для достижения поставленной цели требуется решить следующие задачи:

\begin{itemize}[leftmargin=0.7cm + \labelwidth - \labelsep]
	\item[---] формализовать задачу, данные;
	\item[---] проанализировать типы СУБД;
	\item[---] провести обзор существующих аналогов;
	\item[---] описать структуру базы данных;
	\item[---] создать базу данных с ролевой моделью;
	\item[---] спроектировать интерфейс для доступа к БД;
	\item[---] разработать приложение для взаимодействия с созданной БД;
	\item[---] исследовать влияние использования индексов на время выполнения запросов к базе данных.
\end{itemize}

В результате выполнения работы была разработана база данных для организации работы магазина электроники, а также приложение для взаимодействия с ней.

По результатам исследования, использование индексов для столбцов, по значениям которых часто осуществляется поиск данных, позволяет снизить время выполнения запросов к базе данных.

\pagebreak