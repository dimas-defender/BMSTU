\section*{СПИСОК ИСПОЛЬЗОВАННЫХ ИСТОЧНИКОВ}
\addcontentsline{toc}{section}{СПИСОК ИСПОЛЬЗОВАННЫХ ИСТОЧНИКОВ}

\begingroup
\renewcommand{\section}[2]{}
\begin{thebibliography}{}
	
	\bibitem{ftrace}
	Трассировка ядра с ftrace [Электронный ресурс]. -- Режим доступа: https://habr.com/ru/company/selectel/blog/280322/ (дата обращения: 20.11.2022).
	
	\bibitem{ways}
	Linux Rootkits — Multiple ways to hook syscall [Электронный ресурс]. -- Режим доступа: https://foxtrot-sq.medium.com/linux-rootkits-multiple-ways-to-hook-syscall-s-7001cc02a1e6 (дата обращения: 12.12.2022).
	
	\bibitem{hookftr}
	Перехват функций в ядре Linux с помощью ftrace [Электронный ресурс]. -- Режим доступа: https://habr.com/ru/post/413241/ (дата обращения: 20.12.2022).
	
	\bibitem{docftrace}
	Документация ftrace [Электронный ресурс]. -- Режим доступа: https://www.kernel.org/doc/Documentation/trace/ftrace.txt  (дата обращения: 20.12.2022).
	
	\bibitem{cilurik}
	Цилюрик О.И. Модули ядра Linux. Внутренние механизмы ядра [Электронный ресурс]. -- Режим доступа: http://rus-linux.net/MyLDP/BOOKS/Moduli-yadra-Linux/kern-mod-index.html (дата обращения: 15.12.2022).
	
	\bibitem{docloki}
	Документация к Loki [Электронный ресурс]. -- Режим доступа: https://grafana.com/docs/loki/latest/ (дата обращения: 27.12.2022).
	
	\bibitem{docgraf}
	Документация к Grafana [Электронный ресурс]. -- Режим доступа: https://grafana.com/docs/grafana/latest/ (дата обращения: 27.12.2022).
	
	\bibitem{vscode}
	Документация к Visual Studio Code [Электронный ресурс]. -- Режим доступа: https://code.visualstudio.com/docs (дата обращения: 16.12.2022).
	
	\bibitem{code}
	Hooking or Monitoring System calls in linux using ftrace [Электронный ресурс]. -- Режим доступа: https://nixhacker.com/hooking-syscalls-in-linux-using-ftrace/ (дата обращения: 20.12.2022).
	
\end{thebibliography}
\endgroup

\pagebreak