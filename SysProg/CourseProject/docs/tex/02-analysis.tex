\section{Аналитический раздел}

\subsection{Постановка задачи}

В соответствии с заданием на курсовой проект необходимо разработать загружаемый модуль ядра, позволяющий перехватить функции в ядре ОС Linux.
Данная задача выполняется для перехвата системных вызовов sys\_clone() и\\ sys\_execve().

Для решения поставленной задачи необходимо:

\begin{itemize}[leftmargin=0.7cm + \labelwidth - \labelsep]
	\item[---] проанализировать существующие методы перехвата функций в ядре;
	\item[---] описать алгоритм перехвата функций;
	\item[---] реализовать загружаемый модуль ядра;
	\item[---] обеспечить логирование информации о системных вызовах;
	\item[---] представить собранные данные в графическом виде.
\end{itemize}

\subsection{Методы трассировки ядра}

Под трассировкой \cite{ftrace} понимается получение информации о том, что происходит внутри работающей системы. Для этого используются специальные программные инструменты, регистрирующие события в системе.

Программы-трассировщики могут одновременно отслеживать события как на уровне отдельных приложений, так и на уровне операционной системы. Полученная в ходе трассировки информация может оказаться полезной для диагностики и решения многих системных проблем.

Трассировку иногда сравнивают с логированием. Сходство между этими двумя процедурами действительно есть, но есть и различия.

Во время трассировки записывается информация о событиях, происходящих на низком уровне. Их количество исчисляется сотнями и даже тысячами. В логи же записывается информация о высокоуровневых событиях, которые случаются гораздо реже: например, вход пользователей в систему, ошибки в работе приложений, транзакции в базах данных и другие.

\subsubsection{Модификация таблицы системных вызовов}

В ОС Linux все обработчики системных вызовов расположены в таблице sys\_call\_table. Подмена значений в этой таблице \cite{ways} приводит к смене поведения всей системы. Таким образом, сохранив исходный обработчик и подставив в таблицу собственный, можно перехватить любой системный вызов.

Преимуществами этого подхода являются:

\begin{itemize}[leftmargin=0.7cm + \labelwidth - \labelsep]
	\item[---] полный контроль над любыми системными вызовами;
	\item[---] минимальные накладные расходы, нужно один раз изменить таблицу;
	\item[---] не требуется каких-либо конфигурационных опций в ядре, а значит, поддерживается широкий спектр систем.
\end{itemize}

Недостатками являются:

\begin{itemize}[leftmargin=0.7cm + \labelwidth - \labelsep]
	\item[---] необходимость поиска таблицы системных вызовов, обхода защиты от модификации таблицы, безопасного выполнения замены;
	\item[---] невозможность замены некоторых обработчиков из-за оптимизаций при обработке системных вызовов;
	\item[---] перехватываются только системные вызовы.
\end{itemize}

\subsubsection{Linux Security API}

Linux Security API \cite{hookftr} -- специальный интерфейс, созданный именно для перехвата функций. В критических местах кода ядра расположены вызовы\newline security-функций, которые в свою очередь вызывают коллбеки, установленные security-модулем. Security-модуль может изучать контекст операции и принимать решение о ее разрешении или запрете.

К недостаткам этого подхода можно отнести:

\begin{itemize}[leftmargin=0.7cm + \labelwidth - \labelsep]
	\item[---] security-модули являются частью ядра и не могут быть загружены динамически;
	\item[---] в системе может быть только один security-модуль.
\end{itemize}

Таким образом, для использования Security API необходимо пересобирать ядро Linux.

\subsubsection{Kprobes}

Kprobes -- это метод динамической трассировки, с помощью которого можно прервать выполнение кода ядра в любом месте, вызвать собственный обработчик и по завершении всех необходимых операций вернуться обратно. Обработчики получают доступ к регистрам и могут их изменять. Таким образом, можно получить как мониторинг, так и возможность влиять на дальнейший ход работы.

В ядре есть 3 вида kprobes:

\begin{itemize}[leftmargin=0.7cm + \labelwidth - \labelsep]
	\item[---] kprobes -- «базовая» проба, которая позволяет прервать любое место ядра;
	\item[---] jprobes -- jump probe, вставляется только в начало функции и дает доступ к ее аргументам для обработчика, а также работает через setjmp/longjmp, то есть более легковесна;
	\item[---] kretprobes — return probe, вставляется перед выходом из функции и дает доступ к ее результату.
\end{itemize}

Преимущества, которые дает использование kprobes для перехвата:

\begin{itemize}[leftmargin=0.7cm + \labelwidth - \labelsep]
	\item[---] хорошо задокументированный интерфейс, работа kprobes по возможности оптимизирована;
	\item[---] kprobes реализуются с помощью точек останова (инструкции int3), что позволяет перехватить любое место в ядре, если оно известно.
\end{itemize}

К недостаткам kprobes относятся:

\begin{itemize}[leftmargin=0.7cm + \labelwidth - \labelsep]
	\item[---] для получения аргументов функции или значений локальных переменных надо знать, в каких регистрах или где в стеке они лежат, и извлекать их оттуда;
	\item[---] накладные расходы на расстановку точек останова;
	\item[---] jprobes объявлены устаревшими и вырезаны из современных ядер;
	\item[---] kretprobes необходимо хранить исходный адрес возврата, в случае переполнения буфера с адресами kretprobes будет пропускать срабатывания;
	\item[---] kprobes основывается на прерываниях, поэтому для синхронизации все обработчики выполняются с отключенным вытеснением, что накладывает ограничения на обработчики -- в них нельзя выделять много памяти, заниматься вводом-выводом, спать в таймерах и семафорах.
\end{itemize}

\subsubsection{Kernel tracepoints}

Kernel tracepoints -- это метод трассировки ядра, работающий через статическое инструментирование кода.

В качестве преимуществ можно выделить:

\begin{itemize}[leftmargin=0.7cm + \labelwidth - \labelsep]
	\item[---] минимальные накладные расходы -- необходимо лишь вызвать функцию трассировки в нужном месте;
	\item[---] возможность перехвата всех функций.
\end{itemize}

К недостаткам данного метода относится:

\begin{itemize}[leftmargin=0.7cm + \labelwidth - \labelsep]
	\item[---] отсутствие хорошо задокументированного API;
	\item[---] не работает в модуле, если включен CONFIG\_MODULE\_SIG.
\end{itemize}

\subsubsection{Фреймворк ftrace}

Ftrace \cite{docftrace} -- это фреймворк для трассировки ядра на уровне функций. Ftrace был разработан Стивеном Ростедтом и добавлен в ядро в 2008 году, начиная с версии 2.6.27. Работает ftrace на базе файловой системы debugfs, которая в большинстве современных дистрибутивов Linux смонтирована по умолчанию.

Реализуется ftrace на основе ключей компилятора -pg и -mfentry, которые вставляют в начало каждой функции вызов специальной трассировочной функции mcount() или \_\_fentry\_\_(). Обычно, в пользовательских программах эта возможность компилятора используется профилировщиками, чтобы отслеживать вызовы всех функций. Ядро же использует эти функции для реализации фреймворка ftrace.

Для популярных архитектур доступна оптимизация -- динамический ftrace. Суть в том, что ядро знает расположение всех вызовов mcount() или \_\_fentry\_\_() и на ранних этапах загрузки заменяет их машинный код на nop -- специальную ничего не делающую инструкцию. При включении трассировки в нужные функции вызовы ftrace добавляются обратно. Таким образом, если ftrace не используется, то его влияние на систему минимально.

В качестве преимуществ можно выделить:

\begin{itemize}[leftmargin=0.7cm + \labelwidth - \labelsep]
	\item[---] перехват любой функции;
	\item[---] наличие подробной документации;
	\item[---] перехват совместим с трассировкой, с ядра можно снимать полезные показатели производительности.
\end{itemize}

В качестве недостатка можно выделить требования к конфигурации ядра. Для успешного выполнения перехвата функций с помощью ftrace ядро должно предоставлять целый ряд возможностей:

\begin{itemize}[leftmargin=0.7cm + \labelwidth - \labelsep]
	\item[---] список символов kallsyms для поиска функций;
	\item[---] фреймворк ftrace в целом для выполнения трассировки;
	\item[---] опции ftrace, критически важные для перехвата.
\end{itemize}

Обычно ядра, используемые популярными дистрибутивами, все эти опции содержат, так как они не влияют на производительность и полезны при отладке.

\subsubsection{Сравнение методов трассировки ядра}

Сравнение рассмотренных методов приведено в таблице \ref{methods}.

\begin{table}[H]
	\caption{Сравнение методов перехвата функций}
	\label{methods}
	\small
	\begin{tabular}{|c|c|c|c|c|c|}
		\hline
		& \specialcell{Linux\\Security API} & \specialcell{Модиф. таблицы\\сиcтемных вызовов} & kprobes & \specialcell{Kernel\\tracepoints} & ftrace\\ \hline
		\specialcell{Наличие\\документ. API} & - & - & + & - & + \\ \hline
		\specialcell{Динамическая\\загрузка} & - & + & + & + & + \\ \hline
		\specialcell{Перехват всех\\функций} & + & - & + & + & + \\ \hline
		\specialcell{Любая конфигурация\\ядра} & - & + & + & - & - \\ \hline
	\end{tabular}
\end{table}

\pagebreak

В ходе анализа методов перехвата функций для решения поставленной задачи был выбран фреймворк ftrace, так как он позволяет перехватить любую функцию, может быть динамически загружен в ядро, а также обладает хорошо задокументированным API.

\subsection{Средства визуализации лог-файлов}

Визуализация количества вызовов конкретных функций позволяет наглядно оценить состояние системы.
В данной работе для этих целей был выбран Loki \cite{docloki} -- набор компонентов, которые предоставляют широкие возможности по обработке и анализу поступающих данных. В отличие от других подобных систем Loki основан на идее индексировать только метаданные логов -- labels, a сами логи сжимать рядом.

Loki-стек состоит из трех компонентов: Promtail, Loki, Grafana. Promtail собирает логи, обрабатывает их и отправляет в Loki для хранения.

Grafana \cite{docgraf} -- это платформа с открытым исходным кодом для визуализации, мониторинга и анализа данных. Grafana запрашивает данные из Loki и показывает их. Инструмент позволяет создавать панели, каждая из которых отображает определенные показатели в течение установленного периода времени.
Искать по логам можно в специальном интерфейсе Grafana -- Explorer, для запросов используется язык LogQL.

\subsection*{Выводы}

В результате сравнительного анализа методов перехвата функций был выбран фреймворк ftrace, так как он полностью отвечает требованиям реализации.