\section*{ВВЕДЕНИЕ}
\addcontentsline{toc}{section}{ВВЕДЕНИЕ}

В настоящее время операционная система Linux прочно занимает лидирующее положение в качестве серверной платформы, опережая многие коммерческие разработки. Тем не менее вопросы защиты информационных систем, построенных на базе этой ОС, не перестают быть актуальными. Существует большое количество технических средств, как программных, так и аппаратных, которые позволяют обеспечить безопасность системы. Это средства шифрования данных и сетевого трафика, разграничения прав доступа к информационным ресурсам, защиты электронной почты, веб-серверов, антивирусной защиты.

Один из способов защиты основан на перехвате системных вызовов операционной системы Linux. Этот способ позволяет взять под контроль работу любого приложения и тем самым предотвратить возможные деструктивные действия, которые оно может выполнить. Также перехват системных вызовов может быть использован для обеспечения возможности мониторинга активности в системе.

Данная работа посвящена исследованию способов перехвата системных вызовов с их последующим логированием и представлением собранных данных в графическом виде для наглядного анализа.

\pagebreak