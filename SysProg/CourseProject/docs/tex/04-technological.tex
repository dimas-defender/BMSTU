\section{Технологический раздел}

\subsection{Выбор языка и среды программирования}

Исходный код операционной системы Linux написан на языке C, поэтому для реализации загружаемого модуля выбран язык С.

Visual Studio Code~\cite{vscode} предлагает такой инструмент для разработчиков, как IntelliSense.
Это множество функций редактирования кода, таких как, например, code completion, parameter info, quick info, member lists.
Поэтому в качестве среды программирования была выбрана VS Code.


\subsection{Функция включения перехвата}

Для описания перехватываемых функций используется структура struct ftrace\_hook, которая приведена в листинге \ref{struct}.

\lstinputlisting[
language=C,
firstline=21,
lastline=27,
caption={Структура для описания перехватываемой функции},
label={struct},
style=customc
]
{listings/my_hook.c}

Поля приведенной структуры имеют следующее значение: name -- имя перехватываемой функции, function -- адрес хука, вызываемого вместо перехваченной функции, original -- указатель на место, куда будет записан адрес перехватываемой функции, address -- адрес перехватываемой функции.

Для обеспечения перехвата \cite{code} необходимо заполнить только поля name,\newline function, original. Для удобства описания можно использовать макрос, а все перехватываемые функции собрать в массив, что показано в листинге \ref{array}.

\pagebreak

\lstinputlisting[
language=C,
firstline=247,
lastline=256,
caption={Массив перехватываемых функций},
label={array},
style=customc
]
{listings/my_hook.c}

Найти адрес перехватываемой функции можно с использованием krpobes, его получение показано в листинге \ref{addr}.

\lstinputlisting[
language=C,
caption={Получение адреса перехватываемой функции},
label={addr},
style=customc
]
{listings/addr.c}

Для включения перехвата необходимо проинициализировать структуру ftrace\_ops. В ней обязательным полем является лишь func, указывающая на коллбек, но также необходимо установить некоторые важные флаги. Они предписывают ftrace сохранить и восстановить регистры процессора, содержимое которых может измениться в коллбеке.

Функция включения перехвата и коллбек представлены в листингах \ref{install} и \ref{callback} соответственно.

\lstinputlisting[
language=C,
firstline=65,
lastline=93,
caption={Функция включения перехвата},
label={install},
style=customc
]
{listings/my_hook.c}

\lstinputlisting[
language=C,
firstline=55,
lastline=63,
caption={Коллбек, выполняющий перехват},
label={callback},
style=customc
]
{listings/my_hook.c}

Функция отключения перехвата представлена в листинге \ref{remove}.

\lstinputlisting[
language=C,
firstline=95,
lastline=104,
caption={Функция отключения перехвата},
label={remove},
style=customc
]
{listings/my_hook.c}

\subsection{Хуки для перехватываемых функций}

Порядок и типы аргументов и возвращаемого значения хуков должны соответствовать прототипу системного вызова. В хуках происходит вызов обработчика и его логирование.

Реализации хуков для системных вызовов sys\_clone() и sys\_execve() приведены в листингах \ref{clone} и \ref{exec} соответственно.

\lstinputlisting[
language=C,
firstline=155,
lastline=171,
caption={Реализация hook\_sys\_clone()},
label={clone},
style=customc
]
{listings/my_hook.c}

\lstinputlisting[
language=C,
firstline=207,
lastline=222,
caption={Реализация hook\_sys\_execve()},
label={exec},
style=customc
]
{listings/my_hook.c}

\pagebreak

\subsection{Сборка загружаемого модуля ядра}

Функции инициализации и выхода для загружаемого модуля приведены в листинге \ref{lkm}.

\lstinputlisting[
language=C,
firstline=258,
lastline=275,
caption={Функции инициализации и выхода},
label={lkm},
style=customc
]
{listings/my_hook.c}

Сборка загружаемого модуля ядра осуществляется с помощью make-файла, текст которого приведен в листинге \ref{make}.

\lstinputlisting[
language=C,
caption={Make-файл для сборки модуля},
label={make},
style=customc
]
{listings/Makefile}

\pagebreak